\chapter{Implementierung des Prototypen}

Um das im vorigen Kapitel dargestellte Konzept zu realisieren, lassen sich folgende Hauptaufgaben ausmachen. Im Folgenden wird auf Informationen oder Probleme innerhalb der einzelnen Schritte eingegangen und diese erläutern.

\section{Einschränkungen}
Die Software kann nur funktionieren, wenn garantiert werden kann, dass keine effektiven Sicherheitsmaßnahmen verwendet werden. Folgende Maßnahmen würden das mitlesen der Kommunikation verhindern.
\begin{enumerate}
    \item Nur ohne \ac{HPKP} %\cite{https://tools.ietf.org/html/rfc7469}
    
    Auf dem Server wird ein für den Client ausgestelltes Zertifikat hinterlegt. Der öffentlichen Schlüssel oder der \glqq subjectPublicKeyInfo\grqq{}, ein weiterer Schlüssel mit Zusatzinformationen über die Verschlüsselung, dieses Zertifikats wird auf dem Client abgespeichert. Anschließend wird dieser gespeicherte Schlüssel mit jeder Anfrage an den Server mitgeschickt. Anschließend kann der Server die eingehenden Nachrichten mithilfe der Informationen aus dem Zertifikat verifizieren und sicherstellen das sie vom tatsächlichen Ziel kommt.
    \item Nur wenn der Zertifizierungspfad beim SSL Zertifikat (\emph{trust chain}) nicht geprüft wird.
    
    Bei Verwendung von TLS ist es möglich, den Client prüfen zu lassen, ob der Server das entsprechende Hersteller Zertifikat besitzt. Jedes gültige Zertifikat muss von einer \ac{CA} ausgestellt werden, welche das Vertrauen und eventuell die Identität des Besitzers verifiziert. Jedoch gibt es auch Implementierungen die vom Zertifikat nur die formalen Details und nicht die Herkunft (CA) prüfen.
    Dies ermöglicht, dass die Kommunikation trotz eines manuell (unverifizierbaren) Zertifikats mit den gleichen Details hergestellt werden kann.
\end{enumerate}

Es wurden viele Open Source Bibliotheken verwendet um den Fokus der Arbeit auf die Neuentwicklung und Lösung der Fragestellung zu legen.

Github wurde im Verlauf der Arbeit verwendet da das Unternehmen wie Chacon et al. in dem Buch Pro Git \cite{Chacon2014} beschreiben, der größte Dienstleister für die Bereitstellung von Git Repositories ist. Neben der Speicherung von Quellcode ist Github ebenfalls eine zentrale Plattform für den Austausch und die Zusammenarbeit von Millionen von Entwicklern. Nicht nur open-source sondern auch closed-source Projekte werden dort gespeichert und mithilfe von Problemverfolgung und Codereviews weiterentwickelt. Für die Verwaltung der Informationen verwendet Github das Versionsverwaltungstool Git und baut somit seine Funktionalität um diese Technologie herum.

%%%%%%%%%%%%%%%%%%%%%%%%%%%%
%Test Client/Broker implementieren
%%%%%%%%%%%%%%%%%%%%%%%%%%%%
\section{Client/Broker implementieren}

    %Gewählte Sprache für Client und Broker
    Der Client und Broker, also das \ac{IoT}-Gerät und der Server des Herstellers, besitzt Anforderungen die sich auf die verwendete Programmiersprache auswirken können.
    \begin{itemize}
        \item Die Programmiersprache sollte schnell und mit minimalem Aufwand zu verwenden sein.
        \item Sollte bereits über eine \ac{MQTT} Bibliothek verfügen.
        \item Der Client sollte anpassbar sein.
        \item Sollte auf Linux laufen um Ressourcen zu sparen.
    \end{itemize}
        Anforderungen an die Programmiersprache
        Welche Sprachen wären basierend darauf möglich/naheliegend gewesen
            Warum sind die Entwickelt worden? 
            Was macht die Programmiersprachen aus?
        Warum Python
            Vorteile der Sprache
            Nachteile der Sprache 
            Probleme resultierend aus der Entscheidung
            Vorteile aus der Entscheidung
    
    %Open-Source Bibliotheken Client/Broker
    paho
        Was ist das
        Vorteile
        Probleme
    hbmqtt
        Was ist das
        Vorteile
        Probleme
            
    %Herausforderungen
%%%%%%%%%%%%%%%%%%%%%%%%%%%%
%Implementieren des Proxys (Backend)
%%%%%%%%%%%%%%%%%%%%%%%%%%%%
\section{Proxy Backend implementieren}

    Es wurden verschiedene Programmiersprachen für den Proxy verwendet. Diese wurde entsprechend der Anforderungen und dem Einsatzgebiet abhängig gewählt.
    Die wichtigste Unterscheidung liegt an der Trennung von Backend (Server und Logik Komponenten) und Frontend (Webseite oder User Interface).
    Zuerst wird auf die Sprache für das Backend eingegangen.
    
%%%%%%%%%%%
%Visual Studio
%%%%%%%%%%%
    Als Entwicklungsumgebung für das Backend wurde \ac{VS} 2016 \cite{microsoft_2019} von Microsoft verwendet. Dadurch sind auch die Projektdateien, in dem Repository \glqq MQTT-Proxy\grqq{} \cite{eisenschmidt_2019} gefunden werden können, in einem Visual Studio Projektfile (.sln) definiert.
    
%%%%%%%%%%%
%Anforderungen an die Programmiersprache
%%%%%%%%%%%
    \begin{itemize}
        \item Die Sprache muss verschiedene Bibliotheken mitbringen die Funktionalitäten wie REST-Interface und \ac{MQTT}-Protokoll bereitstellen. Dies ist aufgrund der kurzen Zeit die für diese Arbeit veranschlagt wurde notwendig um alle Features implementiert zu bekommen.
        \item Das Programm muss auf mindestens zwei Plattformen verwendbar sein um den praktischen Nutzen zu steigern.
        \item Es soll eine bekannte Sprache sein, um eine Weiterentwicklung von Dritten zu ermöglichen.
        \item Die Programmiersprache muss dem Autor bereits bekannt sein, um die Einarbeitungszeit minimal zu halten und auch eine entsprechende Qualität zur Verfügung stellen zu können.
    \end{itemize}
%%%%%%%%%%%
%Sprache
%%%%%%%%%%%
    %Welche Sprachen wären basierend darauf möglich/naheliegend gewesen
    Basieren auf diesen Anforderungen stehen Java, C\# und JavaScript zur Auswahl.
    Der Autor hat sich anschließend aus folgenden Gründen für C\# entschieden.
    %Warum sind die Entwickelt worden? / Vorteile Background
    %Was macht die Programmiersprachen aus? / Schwerpunkt
    \begin{itemize}
        \item Assemnly System, nicht linking nötigt, pluginsystem einfacher zu implementieren
        \item Nuget für einfache Dependency Management
        \item Da in Visual Studio gecoded wird
    \end{itemize}
    
    
%%%%%%%%%%%
%Nuget %https://link.springer.com/book/10.1007%2F978-1-4302-4192-8
%%%%%%%%%%%
    NuGet wurde aus den folgenden Vorteilen ausgewählt, welche ebenfalls von Balliauw et al. in seinem Buch Pro NuGet \cite{balliauw2012pro} beschrieben werden.
    Der sogenannte Paket-Manager unterstützt bei der Verwaltung von Abhängigkeiten, wie zum Beispiel Bibliotheken, mit dem Auflösen von Abhängigkeiten und Versionsproblemen. Versionsprobleme treten dann auf, wenn Bibliotheken Abhängigkeiten mit einer Version < X und eine weitere Bibliothek Version > X benötigen. Ein weitere oft auftretender Fall ist, dass beim Aktualisieren einer Bibliothek die benötigte Version für die Abhängigkeiten erhöht aber nicht ebenfalls aktualisiert wird. Die Folge ist eine defekte Abhängigkeit für die Bibliothek und also ein nicht kompilierbarer Zustand  und somit nicht ausführbares Programm. Zusätzlich übernimmt es auch die Installation durch suchen, installieren und aktualisieren oder entfernen von gewünschten Paketen. Ein Weiterer Vorteil ist, dass der Quellcode öffentlich verfügbar ist und das Tool somit weiter angepasst oder verändert werden kann.
    
%%%%%%%%%%%
%Bibliothek
%%%%%%%%%%%
    NewsoftJSON
        Was ist das
        Vorteile
        Probleme
    MQTTNet
        Was ist das
        Vorteile
        %Probleme
        Da die Bibliothek MQTTnet \cite{chkr1011_2018},
        welche in dieser Arbeit verwendet wird, diese Funktionalität nicht besitzt ist es notwendig sie selbst zu implementieren. Um das gleiche Ergebnis wie die der Bridge zu erreichen muss an den Broker eine Client Komponente angeschlossen werden der die Kommunikation nach außen steuert und die Antworten weiter verarbeitet. 
    SharpRest
        Was ist das
        Vorteile
        Probleme
    
%%%%%%%%%%%%%%%%%%%%%%%%%%%%
%Kommunikation mit Firewall
%%%%%%%%%%%%%%%%%%%%%%%%%%%%
\section{Kommunikation steuern}

    %MITM (APR Spoof, DNS, ICMP, FW)
        
%%%%%%%%%%%%%%%%%%%%%%%%%%%%
%Implementieren des Proxys (Frontend)
%%%%%%%%%%%%%%%%%%%%%%%%%%%%
\section{Proxy Frontend implementieren}
    
    Nun wird auf die Programmiersprache für das Frontend, also die Oberfläche eingegangen.
    
%%%%%%%%%%%
%Visual Studio Code
%%%%%%%%%%%
    %https://proquestcombo.safaribooksonline.com/9781484242247
    Zum entwickeln vom Frontend, also die Weboberfläche mit der der Nutzer interagiert und die Applikation steuern kann, wurde Visual Studio Code \cite{microsoft_2016} ebenfalls von Microsoft verwendet.
    
%%%%%%%%%%%
%Anforderungen an die Programmiersprache
%%%%%%%%%%%
    \begin{itemize}
        \item Es muss möglich sein, einzelne Inhalte auf der Webseite dynamisch, also ohne Aktualisierung der Webseite, austauschen zu können.
        \item Wenn neu abgefangenen Nachrichten im Proxy verfügbar sind, sollen sie automatisch an das Frontend weitergegeben werden und das Datenmodel der Oberfläche automatisch aktualisieren.
        \item Die Oberfläche soll aus Komponenten bestehen und mithilfe dieser einen Modularen und strukturierten Aufbau ermöglichen.
        \item Die Eigenschaften des Attributs für das Aussehen der Komponenten darf nicht im gleichen Bereich anderer Komponenten liegen, da sie sich sonnst gegenseitig überschreiben können.
        \item Die Programmiersprache muss dem Autor bereits bekannt sein, um die Einarbeitungszeit minimal zu halten und auch eine entsprechende Qualität zur Verfügung stellen zu können.
    \end{itemize}
    
%%%%%%%%%%%
%Welche Sprachen wären basierend darauf möglich/naheliegend gewesen
%%%%%%%%%%%
    Durch die existierenden Anforderungen hat sich JavaScript als passend herausgestellt. Dadurch, dass verschiedene Frameworks und Erweiterungen auf Basis von JavaScript existieren, werden im Folgenden noch die konkreten Frameworks auf die Anforderungen untersucht.
    %Warum sind die Entwickelt worden? / Vorteile
    %Was macht die Programmiersprache aus? / Schwerpunkt
    
%%%%%%%%%%%
%Warum Vue.js
%%%%%%%%%%%
    Vue.js \cite{you2018vue} wurde ursprünglich für schnelles Prototyping von Webseiten entwickelt, hat sich aber mit der Zeit in die Liste der beliebtesten Webtechnologien eingereiht. Neben Angular und React, stellt es eine Alternative zum schnellen und leichtgewichtigen Entwickeln mit \ac{MVVM}-Architekturmuster dar. Dies ermöglicht die Wiederverwendung von Code durch Nutzung von Komponenten. Durch die bidirektionale Databindung ist es möglich Änderungen im Clientmodel direkt, ohne Aktualisierung, anzeigen zu lassen. In der gegengesetzten Richtung, werden die Informationen, welche zum Beispiel in Inputfeldern bearbeitet werden können, direkt im Clientmodel abgespeichert. Ein weiterer Vorteil ist die direkte Abbildung vom Datenmodel also Datenstruktur zum ViewModel. Dies ermöglicht auch eine Trennung von UI-Designern und Entwicklern da getrennt entwickelt werden kann ohne Informationen in Bereichen der anderen abändern zu müssen.
    %%%%%%%%%%%%%%%%%%%%%%%%%%%%%%%%%%%%%%%
    % Danke an Moritz für die Infos \.A./ %
    %%%%%%%%%%%%%%%%%%%%%%%%%%%%%%%%%%%%%%%
    
%%%%%%%%%%%
%Bibliothek
%%%%%%%%%%%
    Bootstrap-vue
        Was ist das
        Vorteile
        Probleme