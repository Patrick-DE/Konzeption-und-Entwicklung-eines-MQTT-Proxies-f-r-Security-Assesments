\chapter{Implementierung}
\section{Rahmenbedingungen}
    Welche Voraussetzungen sind für das Programm notwendig?
    \begin{itemize}
        \item Betriebssystem: ab Windows 7, Linux (mit Mono*)
        \item Abhängigkeiten: .NetFramework XYZ**
        \item Speicher: 
        \item RAM: 
        \item Prozessor: 
    \end{itemize}
    %Auch nochmal was für Einschränkungen bezüglich der Sicherheitsmechanismen oder nur in Konzept oder nur hier????
    
    %Was ist Mono?
    *Mono ist ein Programm...
    
    **Warum die, was für Konsequenzen würden die haben?

\section{Umgebung und Konfiguration} %Projektierung TODO stimmt das?
    %IDE https://proquestcombo.safaribooksonline.com/9781119404583
    Als Entwicklungsumgebung für das Backend wurde \ac{VS} 2016 \cite{JohnsonBruce2018Pvs2}
    von Microsoft verwendet. Dadurch sind auch die Projektdateien, in dem Repository \glqq MQTT-Proxy\grqq{} \cite{eisenschmidt_2019}
    gefunden werden können, in einem Visual Studio Projektfile (.sln) definiert.
    %Warum VisualStudio
    %\cite{johnson2012professional}
    
    %Konfiguration
    Das Projekt ist im Standard auf x86 Prozessoren ausgerichtet, kann allerdings bei Bedarf auch auf x64 Prozessoren durch Änderung der entsprechenden Einstellung in \ac{VS} geändert werden.
    Eine Unterstützung für weitere Architekturen ist nicht gegeben.
    Zusätzlich ist bei Ausführung der Anwendung ein privilegierter Account erforderlich um die Berichtigung für Zugriff auf die erforderlichen Port zu erhalten.
    Es werden Port 1883 für die Kommunikation mit dem \ac{MQTT}-Protokoll und Port 80 für die \ac{REST} Schnittstelle, welche auch die Weboberfläche bereitstellt, benötigt.
    Der Proxy ist ausschließlich über die Adresse \glqq 127.0.0.1\grqq{} oder \glqq localhost\grqq{} erreichbar, da keine Benutzerauthentifizierung implementiert wurde. Da eine unberechtigte Verwendung im lokalen Netz oder bei Bereitstellung über eine öffentlich erreichbare Adresse nicht ausgeschlossen werden kann, wird die Änderung auch nicht empfohlen. Sofern jedoch Bedarf sein sollte, ist es möglich die Adresse im Quellcode, durch die lokale IP-Adresse des Gerätes auf dem der Proxy läuft, auszutauschen um eine parallele oder externe Verwendung zu ermöglichen.
    
    %https://proquestcombo.safaribooksonline.com/9781484242247
    Für das Frontend, also die Weboberfläche mit der der Nutzer interagiert und die Applikation steuern kann, wurde Visual Studio Code \cite{sole_2019} ebenfalls von Microsoft verwendet.
    %Warum VC 
    
    %Node.js https://ieeexplore.ieee.org/abstract/document/5617064
    Des Weiteren ist wurde durch die Verwendung von Vue.js \cite{you2018vue} auf Basis von Node.js \cite{tilkov2010node} entschieden.

    Nuget \cite{balliauw2012pro} %https://link.springer.com/book/10.1007%2F978-1-4302-4192-8
    
    Github Quellcodeverwaltung %\cite{https://github.com}

\section{Sprachen}
    Gewählte Sprache für ProxyBroker
    Backend
        Anforderungen an die Programmiersprache
        Welche Sprachen wären basierend darauf möglich/naheliegend gewesen
            Warum sind die Entwickelt worden? / Vorteile Background
            Was macht die Programmiersprachen aus? / Schwerpunkt
        Warum C\#
            Vorteile der Sprache
            Nachteile der Sprache 
            Probleme resultierend aus der Entscheidung
            Vorteile aus der Entscheidung
    Frontend
        Anforderungen an die Programmiersprache
        Welche Sprachen wären basierend darauf möglich/naheliegend gewesen
            Warum sind die Entwickelt worden? / Vorteile
            Was macht die Programmiersprachen aus? / Schwerpunkt
        Warum Node.js
            Node.js ist eine JavaScript-Laufzeitumgebung, die auf der JavaScript-Engine V8 des Chrome-Projekts basiert[12]. Die Umgebung sieht asynchrone und ereignisbasierte Programmierung vor: während auf Antwort von langsamen Systemen (bei z.B. Dateioperationen oder Webanfragen) gewartet wird, kann weiterer Code ausgeführt werden [13]. Der zugehörige Paketmanager, npm (Node.js Package Manager), ermöglicht die einfache Installation von einer Vielzahl an frei verfügbaren Open-Source-Bibliotheken. 
        Warum Vue.js
            Vue.js ist eine neben Angular und React sehr populäre Frontendtechnologie, die das MVVM-Architekturmuster (model-view-viewmodel) und Wiederverwendung von Code durch Nutzung von Komponenten einsetzt[14]. In Kombination mit Node.js können Prototypen für Frontends in kurzen Entwicklungszyklen erstellt, erprobt und vorgeführt werden.
            %Eiskalt von moritz geklaut :D
    Gewählte Sprache für Client und Broker
        Anforderungen an die Programmiersprache
        Welche Sprachen wären basierend darauf möglich/naheliegend gewesen
            Warum sind die Entwickelt worden? 
            Was macht die Programmiersprachen aus?
        Warum Python
            Vorteile der Sprache
            Nachteile der Sprache 
            Probleme resultierend aus der Entscheidung
            Vorteile aus der Entscheidung
     
\section{Bibliotheken}
    Wurden viele Open-Source Bibliotheken verwendet um den Fokus der Arbeit auf die Neuentwicklung und Lösung der Fragestellung zu haben.
    Backend
        NewsoftJSON
            Was ist das
            Vorteile
            Probleme
            Alternativen
        MQTTNet
            Was ist das
            Vorteile
            Probleme
            Alternativen
        SharpRest
            Was ist das
            Vorteile
            Probleme
            Alternativen
    Frontend
        Bootstrap-vue
            Was ist das
            Vorteile
            Probleme
            Alternativen
            
    Open-Source Bibliotheken Client/Broker
        paho
            Was ist das
            Vorteile
            Probleme
            Alternativen
        hbmqtt
            Was ist das
            Vorteile
            Probleme
            Alternativen
        
Um dieses Ziel zu erreichen, lassen sich folgende Hauptaufgaben ausmachen.
\begin{enumerate}
    \item Test Client implementieren
    \item Test Broker implementieren
    \item Implementieren des Proxy-Brokers (normale Broker Funktionalität)
    \item Nachrichten des IoT-Clients auf eigenen Broker umleiten
        MITM (APR Spoof, DNS, ICMP, FW)
    \item Implementieren des Proxy-Brokers (weiterleiten an Broker)
    \item Implementieren des Proxy-Brokers (empfangen vom Broker)
    \item Implementieren der Oberfläche zum Steuern
\end{enumerate}