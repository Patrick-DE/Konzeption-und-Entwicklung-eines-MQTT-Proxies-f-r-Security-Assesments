% Die längste Abkürzung kann in die eckigen Klammern
% bei \begin{acronym} geschrieben, um einen hässlichen
% Umbruch zu verhindern
\begin{acronym}[IEEE]
\acro{IoT}{Internet of Things}
\acro{MITM}{Man in the Middle}
\acro{MQTT}{Message Queuing Telemetry Transport}
\acro{OWASP}{Open Web Application Security Project}
\acro{DNS}{Domain Name System}
\acro{DNSSEC}{Domain Name Service Security Extention}
\acro{REST}{Representational State Transfer}
\acro{RAM}{Random Access Memory}
\acro{VMDK}{Virtual Machine Disk}
\acro{VM}{Virtuelle Maschine}
\acro{PTES}{Penetration Testing Execution Standard}
\acro{RAT}{Remote Access Tool}
\acro{ZAP}{Zat Attack Proxy}
\acro{TCP}{Transmission Control Protocol}
\acro{JIT}{Just-in-Time}
\acro{npm}{Node Package Manager}
\acro{MVVM}{model-view-viewmodel}
\acro{HPKP}{HTTP Public Key Pinning}
\acro{CA}{Certification Authority}
\acro{DNS}{Domain Name Server}
\acro{HTML}{Hyper Text Markup Language}
\acro{WWW}{World Wide Web}
\acro{IDE}{Integrated Developement Environment}
\acro{VLAN}{Virtual LAN}
\acro{MB}{Megabyte}
\acro{CPU}{Central Processing Unit}
\acro{M2M}{Machine-to-Machine}
\acro{QoS}{Quality of Service}
\acro{TLS}{Transport Layer Security}
\acro{SSL}{Secure Socket Layer}
\acro{HTTP}{Hyper Text Transport Protocol}
\acro{HTTPS}{Hyper Text Transport Protocol Secure}
\end{acronym}
