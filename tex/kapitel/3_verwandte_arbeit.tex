\chapter{Verwandte Arbeiten}
In diesem Kapitel wird auf zwei Arbeiten eingegangen, welche sich mit dem Thema Sicherheit im Bereich \ac{IoT} beschäftigen. Anschließend werden zwei bereits existierende Lösungen zum Thema weiterleiten / abfangen von Nachrichten des \ac{MQTT} Protokolls genauer betrachtet.

\section{IoT Security}
    \subsection{OWASP IoT Guide}
    Das \ac{OWASP} ist eine offene Gemeinschaft, welche offene und unentgeltliche nutzbare Software zum Testen von IT Sicherheit bereitstellt. Darüber hinaus veröffentlichen sie ebenfalls jährliche Ranglisten bezüglich der am häufigsten vorgefundenen Schwachstellen des vergangenen Jahres. Die Ziele sind vor allem Personen auf die Probleme hinzuweisen, und sie somit zum kritischeren Nachdenken bringen sowie darauf Hinweisen vorsichtiger im Umgang mit digitalen Medien zu sein. Ebenfalls werden Entwickler durch Informationen oft auftretender Schwachstellen und dessen Behebung sowie Methodiken unterstützt die für eine sichere Entwicklung und auch Architektur entscheidend sind.
    
    Aus diesem Zusammenschluss vieler Sicherheitsexperten ist ebenfalls das Manufacturer \ac{IoT} Security Guidance Dokument entstanden. Es beschreibt wie Hersteller von intelligenten Geräten sichere Produkte erstellen können. Es wird den Entwicklern ein Reihe an grundlegenden Richtlinien bereitgestellt, welche mindestens berücksichtigt werden sollten, um die Sicherheit stark zu erhöhen ohne die Kosten ins Unermessliche steigen zu lassen. %cite{https://www.owasp.org/index.php/IoT_Security_Guidance}
    Im Folgenden wird auf ausgewählte Probleme des zuvor genannten Dokuments eingegangen, welche im Rahmen dieser Arbeit thematisiert werden.
    \begin{itemize}
        \item I2:  Unzureichende Authentifizierung
        
        Grundlegend sollte es möglich sein, das vom Hersteller eingestellte Passwort von einem Nutzer abzuändern.
        
        Des Weiteren werden Grundregeln für Passwörter bei der Auslieferung und Änderung empfohlen. Dies soll sicherstellen, dass nicht nur zur Inbetriebnahme sondern auch während der Verwendung des Produktes beim Endkunden der Zugriff zu privilegierten Funktionen und Bereichen unberechtigten effektiv verwehrt wird. Das Bundesamt für Informationssicherheit %cite{https://www.bsi-fuer-buerger.de/BSIFB/DE/Empfehlungen/Passwoerter/passwoerter_node.html} 
        empfiehlt als Minimum die folgenden Regeln.
        \begin{enumerate}
            \item Mindestens 8 Zeichen
            \item Großbuchstaben
            \item Kleinbuchstaben
            \item Zahlen
            \item Sonderzeichen (,.?!=()-...)
            \item Keine Wörter die im Wörterbuch stehen
        \end{enumerate}
        
        Des Weiteren wird eine zwei Faktor Authentifizierung als notwendig angesehen, für den Fall, dass das Passwort doch ausgelesen oder abgefangen wurde. Zwei Faktor bedeutet, dass ein zweiter Weg für die Bestätigung der Identität genutzt wird wie zum Beispiel eine SMS oder Benachrichtigung in einer Applikation über das Handy.
        
        \item I4: Fehlende Transportverschlüsselung
        
        Dies ist der Angriffspunkt, mit dem sich die Arbeit und Erstellung der Software am meisten beschäftigt.
        Der Datenverkehr zwischen den Komponenten sowie den Geräten und dem Ziel sollte verschlüsselt sein um das Mitlesen oder Manipulieren der Nachrichten zu verhindern.
        
        Eine Verschlüsselungen zu verwenden hilft jedoch nicht immer beim erreichen der Sicherheitsziele Integrität, Vertraulichkeit. Nur für den Fall, dass die Verschlüsselung auch noch auf dem aktuelle Stand und noch nicht ausgehebelt wurde. %cite{https://www.bsi.bund.de/SharedDocs/Downloads/DE/BSI/Publikationen/TechnischeRichtlinien/TR02102/BSI-TR-02102.pdf?__blob=publicationFile}
        
        Für die sichere Übertragung steht SSL/TLS zur Verfügung welche verwendet werden sollte um ebenfalls eine Manipulation zu vermeiden. Diese ist ebenfalls im Falle vom \ac{MQTT} Protokoll möglich, jedoch nicht im Standard enthalten wie im Kapitel 2.2.3 erklärt wird.
        
        \item I5: Datenschutzbedenken
        
        Dies ist ein weiterer Teil, welcher mithilfe der hier zu entwickelten Software untersucht werden soll. Die Frage, welche Daten übertragen werden.
        
        Sicherstellen, dass nur die nötigsten personenbezogenen Daten gesammelt und übertragen werden. Diese sollten dann auch anonymisiert werden um keinen Rückschluss auf die Person oder den Account schließend zu können.
        Selbstverständlich dürfen auch nur speziell zugelassene Personen die Daten erheben und übertragen können.
        Des Weiteren spielt auch im Bereich des Datenschutzes spielt die Verschlüsselung eine Rolle, denn die sensiblen Daten sollten zu jeder Zeit verschlüsselt sein.

        \item I8: Unzureichende Anpassungen im Bereich der Sicherheit
        
        Unzureichendes Loggen von Sicherheitsevents wie Angriffen oder Meldung über manipulierte oder unrealistische Nachrichten sind ebenfalls notwendig um rechtzeitig reagieren zu können. Der Nutzer sollte darüber schnellstmöglich informiert werden um entsprechende Maßnahmen einleiten zu können wie Passwörter ändern oder das Gerät vom Netz nehmen.
        
        Mithilfe dieser Maßnahmen wäre es ebenfalls möglich unberechtigte Aktivitäten Dritter schnellstmöglich zu unterbinden.
        
    \end{itemize}
    
    Aus diesen ausgewählten Punkten ergeben sich ebenfalls eine Auswahl an entsprechenden Angriffsvektoren auf \ac{IoT} Geräte. Diese unterstützen, im Gegensatz zu den Herstellern, die Sicherheitsprüfer oder -tester und gelten als strukturierter Leitfaden für die Suche nach Schwachstellen. Ein solches Dokument ist ebenfalls auf der Seite des \ac{OWASP} unter dem Namen "IoT Testing Guides" %\cite{https://www.owasp.org/index.php/IoT_Testing_Guides}
    verfügbar.
    
    \subsection{Internet of things (IoT) security}
    %https://ieeexplore.ieee.org/abstract/document/6978614
    In dem Forschungsartikel "IoT Security: Ongoing Challenges and Research Opportunities" von Z. Zhang et al. \cite{6978614} beschreiben die Autoren das durch den Anstieg in dem Feld nicht nur die Angriffsfläche steigen wird sondern auch neue Angriffsvektoren hinzukommen.
    Sie nennen zwei Sicherheitsprobleme, welche eine entscheidende Rolle in Zukunft spielen werden.
    \begin{enumerate}
        \item Die Geräte
        
        Die Geräte verwenden eine Software, wo die Architektur nicht immer vollständig durchdacht oder mit Fokus auf Sicherheit entwickelt wurde. Dies kann zu Schwachstellen und dadurch Kompromittierungen von Daten der Geräten führen. Bereits gelöste Probleme kommen wieder zum Vorschein, da die Geräte nicht die gleichen Spezifikationen haben wie die, die wir jeden Tag verwenden um uns die Arbeit zu erleichtern. 
    
        Es ist laut einer Umfrage von Statista %cite{https://www.statista.com/statistics/831577/us-usage-of-antivirus-software-by-device/} 
        heutzutage oft der Fall, dass Antivirus Software zum Schutz des Computers oder Smartphone verwendet werden. Diese Software benötigt allerdings viele Ressourcen um verwendet werden zu können.
        Kaspersky %cite{https://support.kaspersky.com/13908}
        als Beispiel, setzt 1150 MB Festplattenspeicher mit einem Intel/AMD 32/64 Bit Prozessor mit 1 GHz und 1 GB freien Arbeitsspeicher nur für die Funktionalität der Software voraus. Vergleicht man das nun mit den Ressourcen, die der intelligenten Sprachsteuerung von Google "Google Home" %\cite{https://de.ifixit.com/Teardown/Google+Home+Teardown/72684}
        zur Verfügung stehen erkennt man eine Diskrepanz in mehreren Punkten. Der Prozessor besitzt eine ARM Architektur, taktet mit 1.2-1.6GHz %\cite{https://developer.arm.com/ip-products/processors/cortex-a/cortex-a7} 
        und ist 64 Bit fähig. Der Prozessor entspricht also nicht nur den Mindestanforderungen der Architektur nicht sondern würde der eigentlichen Anwendung auf dem Gerät eine zu geringe Performance ermöglichen. Des Weiteren entspricht auch der Festplatten-Speicher von 265MB und Arbeitsspeicher 512MB nicht den Mindestanforderungen.
        Somit kommen Sicherheitsprobleme, welche in der Vergangenheit bereits gelöst wurden, erneut zum Vorschein und bedürfen einer neuen Lösung.
        
        Zusätzlich zu den Sicherheitsmechanismen, welche nun nicht mehr verwendet werden können, sind \ac{IoT} Geräte nicht nur von einem Typ. Es gibt viele verschiedene Hersteller und Geräte die sich in den Funktionen, Erscheinungen und Spezifikationen unterscheiden. Diese heterogene Landschaft erhöht die Komplexität, eine Lösungen für alle Geräte zu finden oder den Aufwand für jedes Gerät einen eigenen Sicherheitsmechanismus zu implementieren.
        
        \item Die Kommunikation
        Die heterogene Landschaft beeinflusst jedoch nicht nur die Komplexität auf der Seite der Geräte sondern auch in Bezug auf die Kommunikation.
    
        Es ist denkbar, dass der Wecker mit den Rollläden kommunizieren kann, dieser wiederum die Fenster dazu bringt sich zum Lüften zu öffnen. Automatisch wird die Kaffeemaschine und das Radio angeschaltet damit der Besitzer Kaffee während den neuesten Meldungen genießen kann.
        Nur dieser einzelne Prozess beinhaltet bereits fünf verschiedene Geräte, welche im ersten Moment nichts miteinander zu tun haben. Doch sind alle voneinander abhängig und der Prozess kann durch das manipulieren eines einzelnen Gerätes in der Kette, entweder gestoppt werden oder auch zu einem ungewollten Ergebnis führen.
        
        Ein Problem innerhalb der Kommunikation ist die Identifikation der Geräte. Aktuell wird \ac{DNS} zum Auflösen der Hostnamen auf die dazugehörige IP-Adresse verwendet. Dieses System ist allerdings anfällig gegen Attacken wie DNS cache poisoning oder \ac{MITM} und somit auch nicht sicher.
        DNS cache poisoning bedeutet, dass durch manipulierte DNS Antworten der Zwischenspeicher, welcher die Gegenüberstellung von IP und Hostnamen besitzt verändert wird. Die Folge daraus ist, dass nicht mehr die IP Adresse des legitime Ziels neben dem Namen (z.B. google.de) sondern die IP Adresse des Angreifers steht und somit auf den Angreifer weiterleitet.
        \ac{DNSSEC} wird von der zentralen Registrierungsstelle für die deutsche Domainendung \glqq .de\grqq{} %\cite{https://www.denic.de/wissen/dnssec/}
        als \ac{DNS} Zusatz beschrieben, der verwendet wird um sicherzustellen, dass der Eintrag sowie der Transportweg zwischen der legitimen Adresse und dem DNS-Server geschützt ist und sich kein dritter Akteur einmischen kann.
        
    \end{enumerate}
    %Zusätzliche Quellen%
    %A Survey on the Internet of Things Security: https://ieeexplore.ieee.org/abstract/document/6746513
    %Blockchain for IoT security and privacy: The case study of a smart home: https://ieeexplore.ieee.org/abstract/document/7917634
    %A Critical Analysis on the Security Concerns of Internet of Things (IoT): http://www.pcporoje.com/filedata/592496.pdf
    %Internet of things (IoT) security: Current status, challenges and prospective measures: https://ieeexplore.ieee.org/abstract/document/7412116
    %A Systemic Approach for IoT Security: https://ieeexplore.ieee.org/abstract/document/6569455
    %Security analysis on consumer and industrial IoT devices: https://ieeexplore.ieee.org/abstract/document/7428064
    %Botnets and Internet of Things Security: https://www.computer.org/csdl/magazine/co/2017/02/mco2017020076/13rRUxZRbvu

\section{Implementierungen}
    %\subsection{Confluent MQTT Proxy}
    %Implementierungen auf dem Markt
    %Confluent MQTT Proxy
    %https://www.confluent.io/confluent-mqtt-proxy/
    %https://docs.confluent.io/current/kafka-mqtt/index.html
    %MQTT Proxy provides a scalable and lightweight interface that allows MQTT clients to produce messages %to Apache Kafka® directly, in a Kafka-native way that avoids redundant replication and increased lag.
    %Ist wohl nur ein Addon über das mithilfe von Clients an Kafaka .. kein Ahnnung was das ist Daten %geschickt werden können.
    \subsection{MQTT Bridges}
    
    \subsection{Axway-API-Management-Plus}
    Axway-API-Management-Plus
    https://github.com/Axway-API-Management-Plus/mqtt-proxy
    The MQTT-Proxy itself sits between the MQTT-Provider \& -Consumer and intercepts incoming MQTT-Commands, with the ability to call a REST-API at the API-Gateway. With that, it is for instance possible to validate, that a certain MQTT-Consumer can subscribe to a topic, as the API-Gateway can easily validate the Subscription-Request using a database, another downstream API, whatever.
    Der Aufbau ist relativ ähnlich.
    Alle MQTT Nachrichten werden auf den Proxy weitergeleitet, dort wird mithilfe eines über REST bereitgestellten Regelwerks die Informationen bearbeitet.
    Einschränkungen (gleich):
    - No configurable routes: Only one broker per mqtt-proxy instance
    - NO TLS support
    - No additional TLS options between the client and mqtt-proxy
    Eigenentwicklunge kann aber:
    - HTTP API to publish a MQTT custom crafted message
    - Replay
    - Modifizieren
    - Löschen