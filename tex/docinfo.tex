% -------------------------------------------------------
% Daten für die Arbeit
% Wenn hier alles korrekt eingetragen wurde, wird das Titelblatt
% automatisch generiert. D.h. die Datei titelblatt.tex muss nicht mehr
% angepasst werden.

\newcommand{\hsmasprache}{de} % de oder en für Deutsch oder Englisch
% Für korrekt sortierte Literatureinträge, noch preambel.tex anpassen
% und zwar bei \usepackage[main=ngerman, english]{babel},
% \usepackage[pagebackref=false,german]{hyperref},
% \usepackage[autostyle=true,german=quotes]{csquotes} und bei
% der Literatur die Sortierreihenfolge mit sortlocale=en_US.

% Titel der Arbeit auf Deutsch
\newcommand{\hsmatitelde}{Konzeption und Entwicklung eines MQTT Proxys für Security Assessments}

% Titel der Arbeit auf Englisch
\newcommand{\hsmatitelen}{Concept and development of an MQTT proxy for security assessments}

% Weitere Informationen zur Arbeit
\newcommand{\hsmaort}{Mannheim}    % Ort
\newcommand{\hsmaautorvname}{Patrick} % Vorname(n)
\newcommand{\hsmaautornname}{Eisenschmidt} % Nachname(n)
\newcommand{\hsmadatum}{01.07.2019} % Datum der Abgabe
\newcommand{\hsmajahr}{2019} % Jahr der Abgabe
\newcommand{\hsmafirma}{} % Firma bei der die Arbeit durchgeführt wurde
\newcommand{\hsmabetreuer}{Prof. Dr. Sachar Paulus, Hochschule Mannheim} % Betreuer an der Hochschule
\newcommand{\hsmazweitkorrektor}{Prof. Dr. Martin Damm, Hochschule Mannheim} % Betreuer im Unternehmen oder Zweitkorrektor
\newcommand{\hsmafakultaet}{I} % I für Informatik oder E, S, B, D, M, N, W, V
\newcommand{\hsmastudiengang}{IB} % IB IMB UIB IM MTB (weitere siehe titleblatt.tex)

% Zustimmung zur Veröffentlichung
\setboolean{hsmapublizieren}{true}   % Einer Veröffentlichung wird zugestimmt
\setboolean{hsmasperrvermerk}{false} % Die Arbeit hat keinen Sperrvermerk

% -------------------------------------------------------
% Abstract

% Kurze (maximal halbseitige) Beschreibung, worum es in der Arbeit geht auf Deutsch
%Problem
\newcommand{\hsmaabstractde}{Immer mehr intelligente Geräte werden im häuslichen Umfeld und der Industrie eingesetzt. Dies ist auf die Reduktion der Betriebskosten sowie die Bequemlichkeit von Menschen zurückzuführen. Oft verwenden diese Geräte ein Protokoll, welches im Standard keine Sicherheit garantiert und Hersteller müssen somit eigene Sicherheitsmechanismen implementieren.
Weitere Probleme, neben den Eigenschaften des Protokolls, sind die fehlerhafte Konfiguration und falsche Einrichtung sowie die intransparente Kommunikation der Produkte. Es entstehen nicht nur Schwachstellen die gefunden werden müssen, sondern auch Datenübertragungen die verhindert werden sollten, um den Abfluss von privaten Daten zu verhindern.
%Ziel
Die Arbeit hat zum Ziel, eine Software zu konzipieren und zu entwickeln, die es möglich macht, die Kommunikation zwischen \acs{IoT}-Geräten und dem Betreiber abzufangen und zu modifizieren.
%Was in der Arbeit gemacht wurde
Um Unterstützung beim Überprüfen solcher Geräte zu erhalten, wird ein Konzept entwickelt, welches nicht nur unabhängig von der Plattform, sondern auch von der Sprache und dem untersuchten Protokoll ist.
%Ergebnisse
Anschließend wird dieses Konzept für das \acs{MQTT}-Protokoll implementiert und in einem virtuellen Labor getestet. Die Ergebnisse zeigen, dass die Software in der Lage ist, die Kommunikation abzufangen und zu verändern. Mithilfe von verschiedenen Funktionen, wie dem Modifizieren der Nachrichten, dem erneuten Senden oder neuen Erstellen dieser, ist es möglich die Kommunikation und Endpunkte zu testen.}

% Kurze (maximal halbseitige) Beschreibung, worum es in der Arbeit geht auf Englisch
\newcommand{\hsmaabstracten}{More and more intelligent devices are used in the domestic environment and industry. This is partially due to the reduction of operating costs and the convenience of users. Often, these devices use a protocol that does not guarantee security by design, manufacturers must, therefore, implement their own security mechanisms.
Additional issues, besides the characteristics of the protocol, are the misconfiguration and improper setup as well as the intrasparent communication of the products. Not only vulnerabilities need to be found, but also data transfers that should be blocked to prevent the outflow of private data.
%Ziel
The objective of this bachelor thesis is to write a concept and create a program with the ability to intercept the communication between an \acs{IoT} device and its target.
%Was in der Arbeit gemacht wurde
To support the testing procedure of such devices, a concept is developed which is not only independent of the platform but also of the programming language and the protocol being used.
%Ergebnisse
Afterwards, this concept, with support for the \acs{MQTT}-protocol, is implemented and tested in a virtual lab. The results show that the software is capable of intercepting and changing communication. Using various features, such as modifying the messages, resending, or rebuilding them, it is possible to test the communication and endpoints.}